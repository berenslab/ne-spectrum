\documentclass{beamer}

\usepackage[english]{babel}
\usepackage[noflama,nosectionpages]{beamerthemehsrm}
% \usepackage{unicode-math}
\usepackage{graphicx}
\usepackage[round]{natbib}

\usetheme{hsrm}
\usepackage{media9}

\title{Progress Report}
\subtitle{t-SNE and other Neighborhood Embedding algorithms}
\author{Jan Niklas Böhm}
\institute{University of Tübingen}
\date{\today}

\begin{document}

\maketitle

\section*{Premise}

\begin{frame}{Neighborhood embeddings}
  \begin{block}{\textbf{Definition:} Neighborhood embeddings}
    A function $f: \mathbb R^{n\times n} \to \mathbb R^s$ that
    visualizes a layout of $n$ points (so $s=2$ here).
  \end{block}
\end{frame}

\begin{frame}{Premise}
  \begin{itemize}
  \item Compare UMAP~\citep{mcinnes2018umap} and t-SNE~\citep{maaten2008visualizing}
  \item Compare FA2~\citep{jacomy2014forceatlas2} and t-SNE
  \end{itemize}

  \bigskip\pause They're all based on computing some sort of force equilibrium.
\end{frame}

\begin{frame}[plain]{}
  \vfill
  \centering\includegraphics[width=\linewidth]{../media/ar-spectrum}
  \vfill
\end{frame}

\section*{UMAP vs. t-SNE}

\begin{frame}{Gradients}

  \begin{align}
    \frac{\partial \mathcal L_{\textrm{t-SNE}}(\rho)}{\partial \mathbf y_i} &\sim \overbrace{\sum_j v_{ij}w_{ij}(\mathbf y_i-\mathbf y_j)}^{\text{attraction}} - \overbrace{\frac{n}{\rho Z}\sum_j w_{ij}^2(\mathbf y_i-\mathbf y_j)}^{\text{repulsion}} \label{eq:tsneGradientExagg}\\
    \frac{\partial \mathcal L_{\textrm{UMAP}}(\gamma)}{\partial \mathbf y_i} &\sim \sum_j v_{ij}w_{ij}(\mathbf y_i-\mathbf y_j) - \gamma \sum_j \frac{1}{d_{ij}^2 +\epsilon}w_{ij}(\mathbf y_i-\mathbf y_j) \label{eq:umapGradientGamma}
  \end{align}

  Equations~\ref{eq:tsneGradientExagg} and~\ref{eq:umapGradientGamma}
  are essentially equal if $\gamma={n/\rho Z}$.

  \bigskip\pause Unfortunately, this factor is unknown a priori.
\end{frame}

\begin{frame}{\protect{Can we make assumptions about $n/\rho Z$?}}
  \includegraphics[width=\linewidth]{../media/distance-correlations}

  \bigskip\pause The expression $n/\rho Z$ depends on both $\rho$ and
  $Z$ in a non-obvious fashion.  Especially $Z$---the normalization
  constant---is difficult to predict.

  \bigskip\pause (Setting $\gamma$ to the values indicated here leads
  to unstable results.)
\end{frame}

\begin{frame}{Excursion: Noise-contrastive estimation}
  A probablity distribution can be learned by sampling from another
  (fixed) distribution and creating a supervised learning
  problem~\citep{hastie2009elements}.

  \bigskip\pause This also works for unnormalized distributions by
  additionally learning the normalization
  constant~\citep{gutmann12nce}.

  This has been implemented for t-SNE~\citep{artemenkov2020ncvis}.
\end{frame}

\begin{frame}
  Equation~\ref{eq:umapGradientGamma} was a lie.  Instead the gradient
  uses negative sampling.

  \begin{align}
    \frac{\partial \mathcal L_{\textrm{UMAP}}(\gamma)}{\partial
    \mathbf y_i} &\sim \sum_j v_{ij}w_{ij}(\mathbf y_i-\mathbf y_j) -
                   \gamma \sum_{j\sim U(n)}^\nu \frac{1}{d_{ij}^2 +\epsilon}w_{ij}(\mathbf y_i-\mathbf y_j)
  \end{align}
\end{frame}

\begin{frame}{UMAP: negative sampling}
  Negative sampling does not account for the normalization constant
  and instead does\dots\pause\ nothing?

  \bigskip\pause In general it's hard to analyze and does not stand on
  sound theoretical footing.

  \bigskip\pause But, as the number of negative samples, $\nu$
  increases, the bias should decrease.
\end{frame}

\begin{frame}{Letting $\nu\to n$}
  \includegraphics[width=\linewidth]{../media/umap-repulsion}

  \bigskip\pause This is very inefficient.
\end{frame}

\begin{frame}{Same experiment with Barnes–Hut}
  \includegraphics[width=\linewidth]{../media/elastic-gammas}
\end{frame}

\begin{frame}{Small conclusion}
  Layout created by UMAP is a result of a biased estimator for the
  repulsive forces.
\end{frame}

\section*{FA2 vs. t-SNE}

\begin{frame}{}
  \begin{align}
    \frac{\partial \mathcal L_{\textrm{t-SNE}}(\rho)}{\partial \mathbf y_i} &\sim \overbrace{\sum_j v_{ij}w_{ij}(\mathbf y_i-\mathbf y_j)}^{\text{attraction}} - \overbrace{\frac{n}{\rho Z}\sum_j w_{ij}^2(\mathbf y_i-\mathbf y_j)}^{\text{repulsion}} \nonumber\\
    \frac{\partial \mathcal L_{\textrm{FA2}}}{\partial \mathbf y_i} &= \sum_j v_{ij}(\mathbf y_i-\mathbf y_j) - \sum_j \frac{(h_i+1)(h_j+1)}{d_{ij}^2}(\mathbf y_i-\mathbf y_j) \label{eq:FA2gradient}
  \end{align}

  \bigskip\pause Attractive forces don't decay.
\end{frame}

\begin{frame}{Beyond FA2 \citep{noack2009modularity}}
  \centering\includegraphics[width=.7\linewidth]{../media/misc/mnist_noack.pdf}
\end{frame}

\begin{frame}{Small conclusion}
  Connection still not that well established.
\end{frame}

\section*{Empirical}
\begin{frame}{Fashion MNIST~\citep{xiao2017fashion}}
  \includegraphics[width=\linewidth]{../media/alt/famnist}
\end{frame}
\begin{frame}{Kannada script~\citep{prabhu2019kannada}}
  \includegraphics[width=\linewidth]{../media/alt/kannada}
\end{frame}
\begin{frame}{Kuzushiji Kanjis~\citep{clanuwat2018kuzmnist}}
  \includegraphics[width=\linewidth]{../media/alt/kuzmnist}
\end{frame}
\begin{frame}{Treutlein human~\citep{kanton2019organoid}}
  \includegraphics[width=\linewidth]{../media/treutlein-human}
\end{frame}
\begin{frame}{Treutlein chimp~\citep{kanton2019organoid}}
  \includegraphics[width=\linewidth]{../media/supp/treutlein-chimp}
\end{frame}
\begin{frame}{Treutlein human \citep{kanton2019organoid}}
  \includegraphics[width=\linewidth]{../media/supp/treutlein-human-highk}
\end{frame}
\begin{frame}{Treutlein chimp\citep{kanton2019organoid}}
  \includegraphics[width=\linewidth]{../media/supp/treutlein-chimp-highk}
\end{frame}
\begin{frame}{Hydra \citep{siebert19hydra}}
  \includegraphics[width=\linewidth]{../media/alt/hydra}
\end{frame}
\begin{frame}{Zebrafish \citep{wagner2018zfish}}
  \includegraphics[width=\linewidth]{../media/alt/zfish}
\end{frame}
\begin{frame}{Mouse cortex \citep{tasic2018shared}}
  \includegraphics[width=\linewidth]{../media/alt/tasic}
\end{frame}

\begin{frame}{Open quesitons}
  \begin{itemize}
  \item Formal analysis of negative sampling
  \item Fractional exponents for Noack/FA2
  \item Analytical solutions for t-SNE via Laplacian with a repulsion
    term?
  \end{itemize}
\end{frame}

\begin{frame}{}
  \tiny
  \bibliographystyle{plainnat}
  \bibliography{bibliography}
\end{frame}

\end{document}
